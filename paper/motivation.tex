\documentclass{article} \usepackage{apacite} \usepackage{graphicx} \usepackage{listings}
\usepackage{epigraph} \usepackage{etoolbox} \usepackage{amsmath}
\makeatletter
\patchcmd{\epigraph}{\@epitext{#1}}{\itshape\@epitext{#1}}{}{}
\makeatother \def\signed
#1{{\leavevmode\unskip\nobreak\hfil\penalty50\hskip2em
\hbox{}\nobreak\hfil#1% \parfillskip=0pt \finalhyphendemerits=0
\endgraf}} \newsavebox\mybox \newenvironment{aquote}[1]
{\savebox\mybox{#1}\begin{quote}} {\signed{\usebox\mybox}\end{quote}}
\DeclareGraphicsExtensions{.pdf,.png,.jpg}
% Default margins are too wide all the way around. I reset them here
\setlength{\topmargin}{-.5in} \setlength{\textheight}{9in}
\setlength{\oddsidemargin}{.125in} \setlength{\textwidth}{6in}

\begin{document} \title{Motivated reasoning in minimal groups}
\author{M. H. Tessler, Erik, and Andres} \renewcommand{\today}{Psych 241\\June 8,
2014} \maketitle

\section{Introduction}

\subsection{Motivated reasoning}

\subsection{Sources of motivated reasoning: optimism biases}

\section{Minimal groups}

\section{A computational model}

\section{Experiment 1}

Both Experiment 1 and Experiment 2 attempt to replicate the motivate reasoning observed by Klein and Kunda (1992) in an online setting. Here the general structure of the experiment will be outlined, While the Experiment 2 section will highlight the improvements made upon Experiment 1. 

The basic premise of this paradigm is people will rate the strength of a person at playing a game differetially depending on whether this person is a partner, opponent or an individual unrelated to the present game. Thus, the effect is set to happen before an actual round is played. 

\subsection{Methods}

\subsection{Materials}

include exclusion criteria

\subsection{Procedure}

\subsection{Results}

\section{Experiment 2}

\subsection{Methods}

\subsection{Materials}

include exclusion criteria

\subsection{Procedure}

\subsection{Results}

\section{Discussion}

\end{document}
