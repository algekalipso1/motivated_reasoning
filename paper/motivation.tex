\documentclass{article} \usepackage{apacite} \usepackage{graphicx} \usepackage{listings}
\usepackage{epigraph} \usepackage{etoolbox} \usepackage{amsmath}
\makeatletter
\patchcmd{\epigraph}{\@epitext{#1}}{\itshape\@epitext{#1}}{}{}
\makeatother \def\signed
#1{{\leavevmode\unskip\nobreak\hfil\penalty50\hskip2em
\hbox{}\nobreak\hfil#1% \parfillskip=0pt \finalhyphendemerits=0
\endgraf}} \newsavebox\mybox \newenvironment{aquote}[1]
{\savebox\mybox{#1}\begin{quote}} {\signed{\usebox\mybox}\end{quote}}
\DeclareGraphicsExtensions{.pdf,.png,.jpg}
% Default margins are too wide all the way around. I reset them here
\setlength{\topmargin}{-.5in} \setlength{\textheight}{9in}
\setlength{\oddsidemargin}{.125in} \setlength{\textwidth}{6in}

\begin{document} \title{Motivated reasoning in minimal groups}
\author{M. H. Tessler, Erik, and Andres} \renewcommand{\today}{Psych 241\\June 8,
2014} \maketitle

\section{Introduction}

\subsection{Motivated reasoning}

\subsection{Sources of motivated reasoning: optimism biases}

\section{Minimal groups}

\section{A computational model}

\section{Experiment 1}

Both Experiment 1 and Experiment 2 attempt to replicate the empirical results observed in Klein and Kunda (1992) but in an online setting. The general structure of the experiment will be outlined, and in the Experiment 2 section improvements made upon Experiment 1 will be highlighted.

As explained above, motivated reasonig manifests in the difference in skill level attributed to a player (who we call DW) that is judged based on previous performance depending. Our study incorporates a 2x3 condition design with a conpetition condition and an evidence condition. The competition condition describes the relationship between the study participant and DW, who is either a partner (P), an opponent(O), or a control (C, who is neither a partner nor an opponent). 

Motivated reasoning anticipates that DW will be judged as having a highest ability in the Partner condition, the lowest ability in the Opponent condition, and in-between when DW in the control condition. 

The second experimental condition determines the amount of evidence available about DW.

Mirroring the conditons of Klein and Kunda (1992), we employed two evidence conditions. These conditions are called the high and the low evidence condition. As a cover story, we tell participants that we will show them the performance of



\subsection{Methods}

For our studies we assigned the inicials DW to the target.

Since the effect is set to happen before the actual game is played, the study does not require putting particpants in touch with each other.
We used Amazon Mechanical Turk to recruit 140 participants. Each participant had 

\subsection{Materials}

We hosted the website at http://langcog.stanford.edu/expts/index-clean.html, and linked to it from Mechanical Turk so that participant could navigate an embeded verion of our experiment in the very webpage they accepted to participate in the study.

\subsection{Procedure}

Each participant is assigned to a competition condition and to an evidence condition at random (with equal probabily for each of the permutations). Before they accept to participate, participants are told that the task will take less than 10 minutes (which is the maximum time allocated to complete the task), that they will be compensated with 25 cents for participating (and that they can leave the study anytime if they so desire). Once they accept to particpate, we ask them to write down their inicials (e.g. AGE) and we explain to them that they are about to play two rounds of a \"word game\" with a live partner. The game consists of two teams, each consisting of two persons: a word selector and a word unscrambler. The precise wording is:

\"Each team will have a word selector and a word unscrambler. The word selector will select a word from a list of 6-letter words. The word will appear scrambled, and the word unscrambler will have 10 seconds to figure out the original word. Each team will repeat this process for ten words, and the team with the greater number of correct answers could earn up to 1 dollar, depending on the margin of difference between the two teams.\"

Stating that they can earn up to 1 extra dollar if they win the game is intended to enhance the outcome dependency of the study, to increase motivation and thus enhance the potential for motivated reasoning. 

To enhance the believability of the setting, we display a dynamic \"waiting\" gif for 20 seconds while the server is \"finding players,\" followed by a 4 second wait to \"load the game.\"
 As a cover story to present participants with evidence about DW's previous performance, we tell them that we will show them the results from a previous round in order for them to become familiar with the game. We then show DW's previous unscrambling performance 


DW's relationship with the participant depends on the competition condition, while the number 

\subsection{Results}




\section{Experiment 2}

\subsection{Methods}

\subsection{Materials}

include exclusion criteria

\subsection{Procedure}

\subsection{Results}

\section{Discussion}


Notice that the potential gain in this study is substantially smaller than the potential gain in the original face-to-face version (where the possible bonus for winning was \$50 dollars). This figure needs to be considered in context; the mean participant remuneration is in fact-to-face studies is typically larger than similar online studies by one or two orders of magnitude. Thus, while the absolute value of the possible bonuses is very different, the relative value of the bonuses compared to the standard awards for participation are approximately the same. 

In light of the null results, it is worth examining whether motivated reasoning arises in a manner that depends on the available gains in a given context, or if the phenomenon requires a significance above certain absolute value (e.g. \$20). First, it could be that the small possible bonus (\$1) fails to arise a sufficiently high degree of motivation in the participants for the effect to be elicited. The problem with this view is that the mean self-reported \"motivation to win\" in the compliant participants of the second study is 0.8402576 (+0.04730962, -0.05015323, 95\% bootstrapped confidence interval) and a large minority even chose the ceiling value (1). To confirm that the \$1 bonus is responsible for an enhanced motivation might be investigated in the future. Second, perhaps the self-reported motivaiton is not truely representative of the kind of motivation that leads to motivated reasoning. If this is so, then it would be worth considering alternative ways to measure motivation (such as perseverance in the face of technical problems).

Another In Study 2 participants reported being native English speakers without exception. Thus, while we did not ask about participant's first language in Study 1, we are confident that non-native English speakers are not the cause of the null result. 


\end{document}
